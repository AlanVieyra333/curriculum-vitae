%%%%%%%%%%%%%%%%%%%%%%%%%%%%%%%%%%%%%%%%%
% Twenty Seconds Resume/CV
% LaTeX Template
% Version 1.1 (8/1/17)
%
% This template has been downloaded from:
% http://www.LaTeXTemplates.com
%
% Original author:
% Carmine Spagnuolo (cspagnuolo@unisa.it) with major modifications by 
% Vel (vel@LaTeXTemplates.com)
%
% License:
% The MIT License (see included LICENSE file)
%
%%%%%%%%%%%%%%%%%%%%%%%%%%%%%%%%%%%%%%%%%

%----------------------------------------------------------------------------------------
%	PACKAGES AND OTHER DOCUMENT CONFIGURATIONS
%----------------------------------------------------------------------------------------

\documentclass[letterpaper]{style} % a4paper for A4

%----------------------------------------------------------------------------------------
%	 PERSONAL INFORMATION
%----------------------------------------------------------------------------------------

% If you don't need one or more of the below, just remove the content leaving the command, e.g. \cvnumberphone{}

\profilepic{foto} % Profile picture

\cvname{Alan Fernando\\Rincón Vieyra} % Your name
\cvjobtitle{Computer Systems Engineer} % Job title/career

\cvdate{23-Dec-1995} % Date of birth
\cvnationality{Mexican}
\cvaddress{Del. Miguel Hidalgo, Mexico City, Mexico.} % Short address/location, use \newline if more than 1 line is required
\cvnumberphone{(+52) 5591250145} % Phone number
\cvmail{alan.vieyra376@gmail.com} % Email address
\cvsite{} % Personal website
\cvlinkedin{https://linkedin.com/in/AlanVieyra333} % LinkedIn account
\cvgithub{http://github.com/AlanVieyra333} % Personal website
\cvfb{} % Facebook account

%----------------------------------------------------------------------------------------

\begin{document}

%----------------------------------------------------------------------------------------
%	 ABOUT ME
%----------------------------------------------------------------------------------------

\aboutme{} % To have no About Me section, just remove all the text and leave \aboutme{}

%----------------------------------------------------------------------------------------
%	 SKILLS
%----------------------------------------------------------------------------------------

% Skill bar section, each skill must have a value between 0 an 10 (float)
\skills{
	{Java Spring/9.5},
	{RESTful API/9.5},
	{Angular 8, React.JS/9},
	{Docker, OpenShift, Jenkins/9},
	{Git, Maven/9},
	{PostgreSQL, MySQL, SQL Oracle/8.5},
	{Linux, C++/7}}

% Skill bar section, each skill must have a value between 0 an 10 (float)
\skillsSecond{
	{Python, ML frameworks/6},
	{AWS, GCP, Azure/5},
	{ELK, Splunk/5},
	{Tomcat, WebSphere/4}}

%------------------------------------------------

% Skill text section, each skill must have a value between 0 an 6
% \skillstext{{Drone programming,},{Microcontrollers programming,},{Teamwork,},{Good programming practices.}}

%----------------------------------------------------------------------------------------
%	 LANGUAGES
%----------------------------------------------------------------------------------------

% Languages bar section, each language must have a value between 0 an 10 (float)
\languages{{Spanish - Native/10}, {English - Intermediate/7}}

%----------------------------------------------------------------------------------------

\makeprofile % Print the sidebar

%----------------------------------------------------------------------------------------
%	 EDUCATION
%----------------------------------------------------------------------------------------

\cvsection{Education}

\begin{twenty} % Environment for a list with descriptions
	%\twentyitem{<dates>}{<title>}{<location>}{<description>}
	\twentyitem{2019-2020}{Master of Computer Science  | A.I. | Incomplete}{México}{CINVESTAV - IPN }
	\twentyitem{2014-2018}{Computer Systems Engineering - Intern}{México}{Escuela Superior de Cómputo - Instituto Politécnico Nacional}
	\twentyitem{2010-2014}{Technical Baccalaureate Certificate}{México}{CECyT 9 “Juan de Dios Bátiz” - Instituto Politécnico Nacional}
	\twentyitem{2007-2010}{High school certificate}{México}{Escuela Secundaria Anexa a la Normal Superior}
\end{twenty}

%----------------------------------------------------------------------------------------
%	 COURSES
%----------------------------------------------------------------------------------------

\section{Courses}

\begin{twentyshort} % Environment for a short list with no descriptions
	%\twentyitemshort{<dates>}{<title/description>}
	\twentyitemshort{2018}{Mobile app development with Flutter}{Udemy}
	\twentyitemshort{2018}{Scrum Fundamentals}{SCRUMstudy}
	\twentyitemshort{2016}{Neural networks, Stanford University}{Coursera}
	\twentyitemshort{2015}{Competitive programming and algorithm analysis}{ESCOM - IPN.}
\end{twentyshort}

%----------------------------------------------------------------------------------------
%	 CERTIFICATES
%----------------------------------------------------------------------------------------

\section{Certificates}

\begin{twentyshort}
	\twentyitemshort{2013}{Java SE (\emph{\href{https://www.youracclaim.com/badges/583548b1-06fa-4dbf-889d-419e6c6c2ec1/public_url}{show}})}{Oracle}
\end{twentyshort}

%----------------------------------------------------------------------------------------
%	 AWARDS
%----------------------------------------------------------------------------------------

\section{Achievements}

\begin{twentyshort}
	\twentyitemshort{2019}{Participant of the \textbf{Code Jam} programming competition}{Google}
	\twentyitemshort{2015-2017}{Competing team on stage
		selection of \textbf{ACM-ICPC}\space\space\space\space}{International
		Collegiate Programming Contest}
	\twentyitemshort{2014}{2nd place in the competition Coding Rush}{ITAM}
	\twentyitemshort{2013}{2nd place in the 18th Mexican Olympiad
		Computer Science }{CDMX}
\end{twentyshort}

%----------------------------------------------------------------------------------------
%	 EXPERIENCE
%----------------------------------------------------------------------------------------

\cvsection{Experience}

\cvevent{Infrastructure support engineer}{IDS Comercial S.A. de C.V - Citibanamex}{January 2021 -- July 2022}{México, Remote work}
\begin{itemize}
	\item Pipelines development in \textbf{Jenkins} for CI/CD, with deployment in \textbf{Nexus} repository.
	\item Process automation using \textbf{bash} scripts.
	\item Monitoring using \textbf{Splunk}, and infrastructure support using \textbf{Service Now}.
\end{itemize}

% \cvevent{Full Stack Developer Sr}{IDS Comercial S.A. de C.V - Citibanamex}{January 2021 -- July 2022}{México, Remote work}
% \begin{itemize}
% 	\item Develop web solutions with \textbf{React JS}.
% 	\item Develop microservices with \textbf{Java Spring Boot} and \textbf{Node.js}.
% 	\item Use of databases in \textbf{PostgreSQL}.
% \end{itemize}

\divider

\cvevent{DevOps engineer}{EVERIS BPO MÉXICO, S. DE R.L. DE C.V. - TELCEL}{September 2019 -- December 2020}{Plaza Carzo, CDMX, México}
\begin{itemize}
	\item \textbf{Docker} container building and handling orchestrated by \textbf{OpenShift}.
	\item Infrastructure as code with \textbf{Ansible} to automate the configuration of environments in the \textbf {OpenShift} platform.
	\item Pipelines development in \textbf{Jenkins} for CI/CD, with deployment in a \textbf{Nexus} repository.
	\item Microservices development using \textbf{Spring} and \textbf{Java}.
	\item Anomaly detection in \textbf{Elsatic} using \textbf{Python} and the \textbf{Pandas} library.
	\item Process optimization using C++.
\end{itemize}


%----------------------------------------------------------------------------------------
%	 SECOND PAGE EXAMPLE
%----------------------------------------------------------------------------------------

\newpage % Start a new page

\makeprofileSecond % Print the sidebar

\cvevent{Fullstack Developer Sr}{HITSS CONSULTING S.A. DE C.V. - TELMEX}{June 2018 -- August 2019}{CDMX, México}
\begin{itemize}
	\item Development of new features and bug fixes for a platform for internal Telmex users.
	\item Was developed with \textbf{Java Spring} for backend and \textbf{primefaces} for frontend.
	\item For database, we used \textbf{Oracle} database.
\end{itemize}

%\divider

%\cvevent{Systems Analyst Jr}{Coordinación de Desarrollo Tecnológico – ESCOM - IPN}{January 2018 -- June 2018}{Zacatenco, CDMX, México}
%\begin{itemize}
%	\item Perform the lifting of \textbf{requirements} with the user.
%	\item Perform requirements management, including control and \textbf{documentation}.
%	\item Generate the project deliverables of the \textbf{Analysis} and  \textbf{Design}.
%	\item Diagram handling  \textbf{UML} for modeling the project.
%\end{itemize}

% \divider

% \cvevent{Bachelor's thesis}{Escuela Superior de Cómputo - IPN}{January 2018 -- December 2018}{Zacatenco, CDMX, México}
% \begin{itemize}
% 	\item Development of a prototype for home delivery, programming a drone with the library  Ardupilot en \textbf{C++}, \textbf{microcontrollers} and management of the methodology \textbf{SCRUM}.
% 	\item Development of a web application for user interaction with the drone, using  \textbf{Angular 6} and  \textbf{Python} with  \textbf{Django}.
% \end{itemize}

\divider

\cvevent{Frontend Developer Jr}{Nova Solutions Systems S.A. DE C.V. - MULTIVA}{March 2017 -- May 2018}{Polanco, CDMX, México}
\begin{itemize}
	\item Development of new functionalities for the online banking of the Multiva bank, using \textbf{Angular 2} for frontend development and a \textbf{git} repository hosted on the \textbf{GitLab} platform.
	\item The methodology used was \textbf{SCRUM}.
\end{itemize}

%\section{Other information}

%\subsection{Review}

%Alice approaches Wonderland as an anthropologist, but maintains a strong sense of noblesse oblige that comes with her class status. She has confidence in her social position, education, and the Victorian virtue of good manners. Alice has a feeling of entitlement, particularly when comparing herself to Mabel, whom she declares has a ``poky little house," and no toys. Additionally, she flaunts her limited information base with anyone who will listen and becomes increasingly obsessed with the importance of good manners as she deals with the rude creatures of Wonderland. Alice maintains a superior attitude and behaves with solicitous indulgence toward those she believes are less privileged.

%\section{Other information}

%\subsection{Review}

%Alice approaches Wonderland as an anthropologist, but maintains a strong sense of noblesse oblige that comes with her class status. She has confidence in her social position, education, and the Victorian virtue of good manners. Alice has a feeling of entitlement, particularly when comparing herself to Mabel, whom she declares has a ``poky little house," and no toys. Additionally, she flaunts her limited information base with anyone who will listen and becomes increasingly obsessed with the importance of good manners as she deals with the rude creatures of Wonderland. Alice maintains a superior attitude and behaves with solicitous indulgence toward those she believes are less privileged.

%----------------------------------------------------------------------------------------

\end{document} 
