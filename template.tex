%%%%%%%%%%%%%%%%%%%%%%%%%%%%%%%%%%%%%%%%%
% Twenty Seconds Resume/CV
% LaTeX Template
% Version 1.1 (8/1/17)
%
% This template has been downloaded from:
% http://www.LaTeXTemplates.com
%
% Original author:
% Carmine Spagnuolo (cspagnuolo@unisa.it) with major modifications by 
% Vel (vel@LaTeXTemplates.com)
%
% License:
% The MIT License (see included LICENSE file)
%
%%%%%%%%%%%%%%%%%%%%%%%%%%%%%%%%%%%%%%%%%

%----------------------------------------------------------------------------------------
%	PACKAGES AND OTHER DOCUMENT CONFIGURATIONS
%----------------------------------------------------------------------------------------

\documentclass[letterpaper]{twentysecondcv} % a4paper for A4

%----------------------------------------------------------------------------------------
%	 PERSONAL INFORMATION
%----------------------------------------------------------------------------------------

% If you don't need one or more of the below, just remove the content leaving the command, e.g. \cvnumberphone{}

\profilepic{foto} % Profile picture

\cvname{Alan Fernando\\Rincón Vieyra} % Your name
\cvjobtitle{Analista de Sistemas Jr /\\Desarrollador Front Sr} % Job title/career

\cvdate{} % Date of birth
\cvaddress{Del. Miguel Hidalgo, CDMX, México.} % Short address/location, use \newline if more than 1 line is required
\cvnumberphone{55272898} % Phone number
\cvmail{alan.vieyra376@gmail.com} % Email address
\cvsite{} % Personal website
\cvlinkedin{https://linkedin.com/in/AlanVieyra333} % LinkedIn account
\cvgithub{http://github.com/AlanVieyra333} % Personal website
\cvfb{http://facebook.com/AlanVieyra333} % Facebook account

%----------------------------------------------------------------------------------------

\begin{document}

%----------------------------------------------------------------------------------------
%	 ABOUT ME
%----------------------------------------------------------------------------------------

\aboutme{} % To have no About Me section, just remove all the text and leave \aboutme{}

%----------------------------------------------------------------------------------------
%	 SKILLS
%----------------------------------------------------------------------------------------

% Skill bar section, each skill must have a value between 0 an 10 (float)
\skills{
	{Java/9},
	{Linux (Debian, Arch)/9},
	{Angular, JavaScript, HTML5, CSS3/8.5},
	{Git/8.5},
	{C, C++/8.5},
	{MySQL, SQL Oracle/8},
	{Python, Django/7.5},
	{Aplicaciones Android/7.5},
	{Contenedores Docker, Kubernets/7},
	{Google Cloud Platform/7}}

% Skill bar section, each skill must have a value between 0 an 10 (float)
\skillsSecond{
	{Servidores WebLogic/7},
	{UML, LaTex/6.5},
	{PHP, C\#, Golang/6},
	{Desarrollo de videojuegos - Unity/5.5},
	{Mobile apps multiplataforma - Flutter/5.5},
	{Lisp, TensorFlow, Prolog/5}}

%------------------------------------------------

% Skill text section, each skill must have a value between 0 an 6
\skillstext{{Programación de Drones,},{Programación de Microcontroldores,},{Trabajo en equipo.},{Buenas prácticas de programación.}}

%----------------------------------------------------------------------------------------
%	 LANGUAGES
%----------------------------------------------------------------------------------------

% Languages bar section, each language must have a value between 0 an 10 (float)
\languages{{Inglés/7}}

%----------------------------------------------------------------------------------------

\makeprofile % Print the sidebar

%----------------------------------------------------------------------------------------
%	 EDUCATION
%----------------------------------------------------------------------------------------

\cvsection{Educación}

\begin{twenty} % Environment for a list with descriptions
	%\twentyitem{<dates>}{<title>}{<location>}{<description>}
	\twentyitem{Desde 2014}{Estudiante de Ingeniería en Sistemas Computacionales - 9°}{México}{\emph{Escuela Superior de Cómputo - Instituto Politécnico Nacional}}
	\twentyitem{2010-2014}{Certificado de bachillerato técnico}{México}{\emph{CECyT 9 “Juan de Dios Bátiz” - Instituto Politécnico Nacional}}
	\twentyitem{2007-2010}{Certificado de secundaria}{México}{\emph{Escuela Secundaria Anexa a la Normal Superior}}
\end{twenty}

%----------------------------------------------------------------------------------------
%	 COURSES
%----------------------------------------------------------------------------------------

\section{Cursos}

\begin{twentyshort} % Environment for a short list with no descriptions
	%\twentyitemshort{<dates>}{<title/description>}
	\twentyitemshort{2018}{After Effects: De zero a Master, Udemy.}
	\twentyitemshort{2018}{Programación de drones usando Ardupilot, CINVESTAV - IPN.}
	\twentyitemshort{2017}{Fundamentos de Inteligencia Artificial, CIC - IPN}
	\twentyitemshort{2017}{Unity, Youtube}
	\twentyitemshort{2016}{Redes neuronales, Stanford University, Coursera}
	\twentyitemshort{2015}{Aplicaciones para dispositivos Android, ESCOM - IPN.}
	\twentyitemshort{2015}{Algoritmia, ESCOM - IPN.}
\end{twentyshort}

%----------------------------------------------------------------------------------------
%	 CERTIFICATES
%----------------------------------------------------------------------------------------

\section{Certificados}

\begin{twentyshort}
	\twentyitemshort{2018}{Scrum Fundamentals, SCRUMstudy.}
	\twentyitemshort{2013}{Java SE, Oracle. \emph{\href{https://drive.google.com/file/d/0B7Bdg32oxtvLTzB1aHhvU3ZQWDA/view}{Ver}}}
\end{twentyshort}

%----------------------------------------------------------------------------------------
%	 AWARDS
%----------------------------------------------------------------------------------------

\section{Logros}

\begin{twentyshort}
	\twentyitemshort{2015-2017}{Equipo competidor en la etapa de
		selección del ACM-ICPC (International
		Collegiate Programming Contest).}
	\twentyitemshort{2014}{Integrante del mejor equipo de Nivel
		Medio Superior en el 6to. Concurso
		Anual de Programación en el marco de
		la XIX EXPO-ESCOM.}
	\twentyitemshort{2014}{2° lugar en la competencia Coding Rush,
		ITAM.}
	\twentyitemshort{2013}{2° lugar en la 18° Olimpiada Mexicana
		de Informática, D.F.}
\end{twentyshort}

%----------------------------------------------------------------------------------------
%	 EXPERIENCE
%----------------------------------------------------------------------------------------

\cvsection{Experiencia}

\cvevent{Analista de Sistemas Jr}{Hitss Solutions, S.A. de C.V.}{Jun 2018 -- Actualidad}{CDMX, México}
\begin{itemize}
	\item Brindar soluciones Web a nuestro principal cliente \textbf{TELMEX} usando diversas herramientas de software.
	\item Uso del framework \textbf{PrimeFaces} para el desarrollo Frontend de la aplicación.
	\item Manejo de Beans, JSF y JPA en Java para el desarrollo del servidor.
	\item Uso de MySQL y Oracle SQL para la manipulación de base de datos usando JOINS, prodecimientos almacenados así como triggers.
\end{itemize}

\divider

\cvevent{Project Manager Jr / Desarrollador Front Sr}{Trabajo Terminal - IPN}{Jun 2017 -- Actualidad}{CDMX, México}
\begin{itemize}
	\item Desarrollo de un prototipo para paquetería a domicilio, programando un drone con Ardupilot uando la metodología SCRUM.
	\item Desarrollo de una web app para comunicar al cliente con el drone. Usandp Angular 6 y Django.
\end{itemize}

%----------------------------------------------------------------------------------------
%	 SECOND PAGE EXAMPLE
%----------------------------------------------------------------------------------------

\newpage % Start a new page

\makeprofileSecond % Print the sidebar

\cvevent{Analista de Sistemas Jr}{Coordinación de Desarrollo Tecnológico - IPN}{Ene 2018 -- May 2018}{CDMX, México}
\begin{itemize}
	\item Realizar el levantamiento de \textbf{requerimientos} con el usuario.
	\item Realizar la gestión de requerimientos, incluyendo el control y documentación de requerimientos.
	\item Generar los entregables del proyecto de la fase de Análisis y Diseño de las solicitudes acordadas.
	\item Manejo de diagramas \textbf{UML} para el modelado del proyecto.
\end{itemize}

\divider

\cvevent{Desarrollador Front Jr}{NOVA SOLUTIONS SYSTEMS S.A. DE C.V.}{Jun 2017 -- Dic 2017}{CDMX, México}
\begin{itemize}
	\item Desarrollo de la banca electrónica del banco \textbf{Multiva}.
	\item Realizar módulos y componentes de la interfaz de usuario en una aplicación web usando \textbf{Angular} como herramienta de desarrollo.
	\item Manejo de \textbf{Git} para el control de versiones y trabajo en equipo.
\end{itemize}

%\section{Other information}

%\subsection{Review}

%Alice approaches Wonderland as an anthropologist, but maintains a strong sense of noblesse oblige that comes with her class status. She has confidence in her social position, education, and the Victorian virtue of good manners. Alice has a feeling of entitlement, particularly when comparing herself to Mabel, whom she declares has a ``poky little house," and no toys. Additionally, she flaunts her limited information base with anyone who will listen and becomes increasingly obsessed with the importance of good manners as she deals with the rude creatures of Wonderland. Alice maintains a superior attitude and behaves with solicitous indulgence toward those she believes are less privileged.

%\section{Other information}

%\subsection{Review}

%Alice approaches Wonderland as an anthropologist, but maintains a strong sense of noblesse oblige that comes with her class status. She has confidence in her social position, education, and the Victorian virtue of good manners. Alice has a feeling of entitlement, particularly when comparing herself to Mabel, whom she declares has a ``poky little house," and no toys. Additionally, she flaunts her limited information base with anyone who will listen and becomes increasingly obsessed with the importance of good manners as she deals with the rude creatures of Wonderland. Alice maintains a superior attitude and behaves with solicitous indulgence toward those she believes are less privileged.

%----------------------------------------------------------------------------------------

\end{document} 
