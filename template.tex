%%%%%%%%%%%%%%%%%%%%%%%%%%%%%%%%%%%%%%%%%
% Twenty Seconds Resume/CV
% LaTeX Template
% Version 1.1 (8/1/17)
%
% This template has been downloaded from:
% http://www.LaTeXTemplates.com
%
% Original author:
% Carmine Spagnuolo (cspagnuolo@unisa.it) with major modifications by 
% Vel (vel@LaTeXTemplates.com)
%
% License:
% The MIT License (see included LICENSE file)
%
%%%%%%%%%%%%%%%%%%%%%%%%%%%%%%%%%%%%%%%%%

%----------------------------------------------------------------------------------------
%	PACKAGES AND OTHER DOCUMENT CONFIGURATIONS
%----------------------------------------------------------------------------------------

\documentclass[letterpaper]{twentysecondcv} % a4paper for A4

%----------------------------------------------------------------------------------------
%	 PERSONAL INFORMATION
%----------------------------------------------------------------------------------------

% If you don't need one or more of the below, just remove the content leaving the command, e.g. \cvnumberphone{}

\profilepic{foto} % Profile picture

\cvname{Alan Fernando\\Rincón Vieyra} % Your name
\cvjobtitle{Computer Systems Engineer} % Job title/career

\cvdate{} % Date of birth
\cvaddress{Del. Miguel Hidalgo, CDMX, Mexico.} % Short address/location, use \newline if more than 1 line is required
\cvnumberphone{(+52) 5591250145} % Phone number
\cvmail{alan.vieyra376@gmail.com} % Email address
\cvsite{} % Personal website
\cvlinkedin{https://linkedin.com/in/AlanVieyra333} % LinkedIn account
\cvgithub{http://github.com/AlanVieyra333} % Personal website
\cvfb{http://facebook.com/AlanVieyra333} % Facebook account

%----------------------------------------------------------------------------------------

\begin{document}

%----------------------------------------------------------------------------------------
%	 ABOUT ME
%----------------------------------------------------------------------------------------

\aboutme{} % To have no About Me section, just remove all the text and leave \aboutme{}

%----------------------------------------------------------------------------------------
%	 SKILLS
%----------------------------------------------------------------------------------------

% Skill bar section, each skill must have a value between 0 an 10 (float)
\skills{
	{Java, C++/9.5},
	{Linux (Debian, Centos, RHEL, Arch)/9.5},
	{Docker, OpenShift/9.5},
	{Angular 8, JavaScript, HTML5, CSS3/9.5},
	{Git, Web Services Rest y SOAP/9.5},
	{Jenkins, Maven, Ansible, Shell/9},
	{MySQL, SQL Oracle/8.5},
	{Mobile apps Android, iOS/8},
	{Mobile apps multiplatform - Flutter/8}}

% Skill bar section, each skill must have a value between 0 an 10 (float)
\skillsSecond{
	{Python, Django/8},
	{AWS, Google Cloud Platform/7.5},
	{Servers WebLogic and WebSphere/6.5},
	{UML, LaTex/6.5},
	{PHP, C\#, Golang/6},
	{ Video games development- Unity/5},
	{Lisp, TensorFlow, Prolog/4}}

%------------------------------------------------

% Skill text section, each skill must have a value between 0 an 6
\skillstext{{Drone programming,},{Microcontrollers programming,},{Teamwork,},{Good programming practices.}}

%----------------------------------------------------------------------------------------
%	 LANGUAGES
%----------------------------------------------------------------------------------------

% Languages bar section, each language must have a value between 0 an 10 (float)
\languages{{Spanish - Native/10}, {English - Intermediate/7}}

%----------------------------------------------------------------------------------------

\makeprofile % Print the sidebar

%----------------------------------------------------------------------------------------
%	 EDUCATION
%----------------------------------------------------------------------------------------

\cvsection{Education}

\begin{twenty} % Environment for a list with descriptions
	%\twentyitem{<dates>}{<title>}{<location>}{<description>}
	\twentyitem{2019-2020}{Master of Computer Science  | Incomplete}{México}{CINVESTAV - IPN }
	\twentyitem{2014-2018}{Computer Systems Engineering - Intern}{México}{Escuela Superior de Cómputo - Instituto Politécnico Nacional}
	\twentyitem{2010-2014}{Technical Baccalaureate Certificate}{México}{CECyT 9 “Juan de Dios Bátiz” - Instituto Politécnico Nacional}
	\twentyitem{2007-2010}{High school certificate}{México}{Escuela Secundaria Anexa a la Normal Superior}
\end{twenty}

%----------------------------------------------------------------------------------------
%	 COURSES
%----------------------------------------------------------------------------------------

\section{Courses}

\begin{twentyshort} % Environment for a short list with no descriptions
	%\twentyitemshort{<dates>}{<title/description>}
	\twentyitemshort{2018}{Mobile app development with Flutter}{Udemy}
	\twentyitemshort{2018}{Scrum Fundamentals}{SCRUMstudy}
	\twentyitemshort{2017}{Unity}{Youtube}
	\twentyitemshort{2016}{Neural networks, Stanford University}{Coursera}
	\twentyitemshort{2015}{Apps for Android devices}{ESCOM - IPN.}
	\twentyitemshort{2015}{Competitive programming and algorithm analysis}{ESCOM - IPN.}
\end{twentyshort}

%----------------------------------------------------------------------------------------
%	 CERTIFICATES
%----------------------------------------------------------------------------------------

\section{Certificates}

\begin{twentyshort}
	\twentyitemshort{2013}{Java SE (\emph{\href{https://drive.google.com/file/d/0B7Bdg32oxtvLTzB1aHhvU3ZQWDA/view}{show}})}{Oracle}
\end{twentyshort}

%----------------------------------------------------------------------------------------
%	 AWARDS
%----------------------------------------------------------------------------------------

\section{Achievements}

\begin{twentyshort}
	\twentyitemshort{2015-2017}{Competing team on stage
		selection of ACM-ICPC\space\space\space\space}{International
		Collegiate Programming Contest}
	\twentyitemshort{2014}{Member of the best level team
		Upper Middle in 6th. Competition
		Annual Programming within the framework of
		the XIX}{EXPO-ESCOM}
	\twentyitemshort{2014}{2nd place in the competition Coding Rush}{ITAM}
	\twentyitemshort{2013}{2nd place in the 18th Mexican Olympiad
		Computer Science }{CDMX}
\end{twentyshort}

%----------------------------------------------------------------------------------------
%	 EXPERIENCE
%----------------------------------------------------------------------------------------

\cvsection{Experience}

\cvevent{DevOps engineer semi Sr}{Everis, S.A. de C.V. - TELCEL}{August 2019 -- Present}{Plaza Carzo, CDMX, México}
\begin{itemize}
	\item \textbf{Docker} container building and handling orchestrated by \textbf{OpenShift}.
	\item Infrastructure as code with \textbf{Ansible} to automate the configuration of environments in the \textbf {OpenShift} platform.
	\item Pipelines development in \textbf{Jenkins} for CI/CD, with deployment in a \textbf{Nexus} repository.
\end{itemize}

\divider

\cvevent{Architect ESB}{Hitss Solutions, S.A. de C.V. - TELCEL}{March 2019 -- August 2019}{Plaza Carzo, CDMX, México}
\begin{itemize}
	\item Development of integration services between applications of the various systems of the organization using the tool \textbf{IBM Integration Bus}, \textbf{Websphere MQ} and communication protocols \textbf{SOAP} y \textbf{REST}.
	\item Container handling \textbf{Docker} orchestrated with \textbf{OpenShift}.
	\item Management \textbf {Jenkins} y \textbf{Maven} for continuous integration.
\end{itemize}

%----------------------------------------------------------------------------------------
%	 SECOND PAGE EXAMPLE
%----------------------------------------------------------------------------------------

\newpage % Start a new page

\makeprofileSecond % Print the sidebar

\cvevent{Systems Analyst Sr}{Hitss Solutions, S.A. de C.V. - TELMEX}{June 2018 -- March 2019}{Coyoacán, CDMX, México}
\begin{itemize}
	\item Development of a mobile application for \textbf{iOS} using \textbf{Swift}.
	\item Development of web applications with use of: \textbf{PrimeFaces}, \textbf{Spring}, \textbf{JSF}, \textbf{JPA}, \textbf{DAO}  and servers \textbf{WebLogic} and \textbf{Tomcat}.
	\item Management \textbf{MySQL} y \textbf{Oracle SQL} for database manipulation using JOINS, stored procedures as well as triggers.
\end{itemize}

\divider

\cvevent{Systems Analyst Jr}{Technological Development Coordination - IPN}{January 2018 -- June 2018}{Zacatenco, CDMX, México}
\begin{itemize}
	\item Perform the lifting of \textbf{requirements} with the user.
	\item Perform requirements management, including control and \textbf{documentation}.
	\item Generate the project deliverables of the \textbf{Analysis} and  \textbf{Design}.
	\item Diagram handling  \textbf{UML} for modeling the project.
\end{itemize}

\divider

\cvevent{Bachelor's thesis}{Instituto Politécnico Nacional (IPN)}{January 2018 -- December 2018}{Zacatenco, CDMX, México}
\begin{itemize}
	\item Development of a prototype for home delivery, programming a drone with the library  Ardupilot en \textbf{C++}, \textbf{microcontrollers} and management of the methodology \textbf{SCRUM}.
	\item Development of a web application for user interaction with the drone, using  \textbf{Angular 6} and  \textbf{Python} with  \textbf{Django}.
\end{itemize}

\divider

\cvevent{Developer Front Jr}{Nova Solutions Systems S.A. DE C.V. - MULTIVA}{March 2017 -- December 2017}{Polanco, CDMX, México}
\begin{itemize}
	\item Development of the bank's electronic banking \textbf{Multiva}.
	\item Make modules and components for the web application using \textbf{Angular} as a development tool, as well as \textbf{HTML5}, \textbf{CSS3} y \textbf{JavaScript}.
	\item Manejo de \textbf{Git} for version control and teamwork.
\end{itemize}

%\section{Other information}

%\subsection{Review}

%Alice approaches Wonderland as an anthropologist, but maintains a strong sense of noblesse oblige that comes with her class status. She has confidence in her social position, education, and the Victorian virtue of good manners. Alice has a feeling of entitlement, particularly when comparing herself to Mabel, whom she declares has a ``poky little house," and no toys. Additionally, she flaunts her limited information base with anyone who will listen and becomes increasingly obsessed with the importance of good manners as she deals with the rude creatures of Wonderland. Alice maintains a superior attitude and behaves with solicitous indulgence toward those she believes are less privileged.

%\section{Other information}

%\subsection{Review}

%Alice approaches Wonderland as an anthropologist, but maintains a strong sense of noblesse oblige that comes with her class status. She has confidence in her social position, education, and the Victorian virtue of good manners. Alice has a feeling of entitlement, particularly when comparing herself to Mabel, whom she declares has a ``poky little house," and no toys. Additionally, she flaunts her limited information base with anyone who will listen and becomes increasingly obsessed with the importance of good manners as she deals with the rude creatures of Wonderland. Alice maintains a superior attitude and behaves with solicitous indulgence toward those she believes are less privileged.

%----------------------------------------------------------------------------------------

\end{document} 
